\section{Implementation} \label{section: Implementation}

The following section describes the general outline of how the algorithms are implemented.

\subsection{Routing algorithm}
The algorithm is implemented as follows:

\begin{algorithm}
    \caption{Placeholder for Algorithm}\label{alg:gen}
    \begin{algorithmic}
        % \Require Some data input
        \State Initialise population list $p$ of size $n_p$ with randomly generated solutions
        \State Initialise empty list $c$ of size $n_c$
        \State Initialise best solution $a = 0$
        \State Initialise iterator $i = 0$
        \State Initialise maximum number of repetitions $r$
        \State Initialise minimum subset size $t$
        \For{parent in list $p$}
        \State Create subset $q$ of population of random size $s\in [t, n_p]$
        \For{element in subset $q$}
        \State Evaluate value of element
        \EndFor
        \State Select two best elements ${e_1, e_2}$ with most weight
        \State Create child $k$ using \textbf{cross-over} of ${e_1}$ and ${e_2}$
        \State \textbf{Mutate} child $k$ \Comment{The mutation adds or removes a pentomino}
        \State Add child $k$ to list $c$
        \If{$n_c$ equals $n_p$}
        \State List $p$ is set to list $c$
        \State List $c$ is emptied
        \State Increase value of iterator $i$ by $1$
        \If{evaluation of $k > a$}
        \State Solution $a$ is set to evaluation of $k$;
        \EndIf
        \If{$i = r$} \Comment{The number $m$ is arbitrarily selected}
        \State break for loop;
        \EndIf
        \EndIf
        \EndFor
        \Ensure $k$
    \end{algorithmic}
\end{algorithm}


Algorithm description goes here.
